\DeclareRobustCommand{\eg}{e.g.,\@\xspace}
\DeclareRobustCommand{\ie}{i.e.,\@\xspace}
\DeclareRobustCommand{\wrt}{w.r.t.\@\xspace}
\DeclareRobustCommand{\wp}{w.p.\@\xspace}
\DeclareRobustCommand{\st}{s.t.\@\xspace}


\newtheorem{problem}{Problem}
\newtheorem{lemma}{Lemma}
\newtheorem{theorem}[lemma]{Theorem}
\newtheorem{corollary}[lemma]{Corollary}
\newtheorem{proposition}[lemma]{Proposition}
\newtheorem{definition}[lemma]{Definition}
\newtheorem{property}{Property}
\newtheorem{assumption}{Assumption}
\newtheorem*{exercise}{Exercise}
\newtheorem*{remark}{Remark}
\newtheorem*{observation}{Observation}
\newtheorem*{example}{Example}
\newtheorem*{solution}{Solution}
\newtheorem*{notation}{Notation}
\newtheorem*{question}{Question}

% 空行
\newcommand\blankline{~\\}
\newcommand\tto{\Rightarrow}

%算法中用\require, \ensure写input,output
\renewcommand{\algorithmicrequire}{\textbf{Input:}} % Use Input in the format of Algorithm
\renewcommand{\algorithmicensure}{\textbf{Output:}} % Use Output in the format of Algorithm
\newcommand{\algorithmiclastcon}{\textbf{Initialize:}}
\newcommand{\Initialize}{\item[\algorithmiclastcon]}


%设置不缩进
\setlength{\parindent}{0pt}

%设置图表的名字
\renewcommand\tablename{Table}
\renewcommand\figurename{Figure}

%设置日期为英文
\CTEXoptions[today=old]

% 设置参考文献为英文
\renewcommand\refname{Reference}

% 设置答题时的题目模版
\renewcommand{\question}[1]{\subsubsection*{Question:#1}}

% 改变行距
\linespread{1}


\renewcommand{\c}[1]{\mathcal{#1}}
\renewcommand{\b}[1]{\mathbb{#1}}
\newcommand{\h}[1]{\hat{#1}}
\newcommand{\s}[1]{\sqrt{#1}}
\renewcommand{\o}[1]{\overline{#1}}
\renewcommand{\u}[1]{\underline{#1}}
\renewcommand{\f}[1]{\frac{#1}{#2}}
